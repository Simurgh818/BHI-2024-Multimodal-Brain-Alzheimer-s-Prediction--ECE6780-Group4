\section{Conclusion: Advancements in Multimodal Brain Data Analysis for Alzheimer’s Disease Prediction.}
Our study introduces a novel approach to Alzheimer’s Disease (AD) classification, leveraging the complementary strengths of Convolutional Neural Networks (CNN) and Vision Transformers (ViT). By combining CNN's proficiency in capturing local features with ViT's ability to model global dependencies, our hybrid model achieves enhanced performance and stability in AD classification tasks. Notably, the hybrid ViT-CNN model demonstrates superior consistency and generalization compared to individual models. Furthermore, the incorporation of a pre-trained ViT significantly improves sensitivity, highlighting its efficacy in facilitating multi-modality integration. While PET emerges as a strong performer in single modality settings, multimodal modeling, particularly combining MRI and PET, significantly enhances classification accuracy and stability. However, the contribution of EHR and genomic data remains limited, underscoring the need for larger and more comprehensive datasets. Our findings emphasize the importance of multimodal approaches in AD diagnosis and suggest avenues for future research to optimize the integration of diverse data modalities for more precise and robust models. 