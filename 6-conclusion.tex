\section{Conclusion: Advancements in Multimodal Brain Data Analysis for Alzheimer’s Disease Prediction.}
Our study presents a novel approach to Alzheimer’s Disease (AD) classification by harnessing the complementary strengths of Convolutional Neural Networks (CNN) and Vision Transformers (ViT). The hybrid ViT-CNN model capitalizes on CNN's capability to capture local features and ViT's strength in modeling global dependencies, resulting in improved performance and stability in AD classification tasks. The hybrid model exhibits superior consistency and generalization compared to individual models, and the incorporation of a pre-trained ViT significantly enhances sensitivity, underscoring its effectiveness in multimodal data integration.

While PET proved to be the most effective single modality, the combination of MRI and PET in a multimodal framework yielded even greater improvements in classification accuracy and robustness. However, the contribution of Electronic Health Records (EHR) and genomic data was limited, indicating the need for larger and more diverse datasets to fully realize the potential of these modalities in AD diagnosis.

Overall, our findings highlight the critical role of multimodal approaches in advancing AD classification and suggest promising directions for future research. These include optimizing the integration of diverse data modalities, improving dataset quality and size, and refining transfer learning techniques to develop more precise, stable, and interpretable models for early AD diagnosis and progression prediction.